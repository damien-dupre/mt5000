% Options for packages loaded elsewhere
\PassOptionsToPackage{unicode}{hyperref}
\PassOptionsToPackage{hyphens}{url}
%
\documentclass[
]{article}
\usepackage{lmodern}
\usepackage{amsmath}
\usepackage{ifxetex,ifluatex}
\ifnum 0\ifxetex 1\fi\ifluatex 1\fi=0 % if pdftex
  \usepackage[T1]{fontenc}
  \usepackage[utf8]{inputenc}
  \usepackage{textcomp} % provide euro and other symbols
  \usepackage{amssymb}
\else % if luatex or xetex
  \usepackage{unicode-math}
  \defaultfontfeatures{Scale=MatchLowercase}
  \defaultfontfeatures[\rmfamily]{Ligatures=TeX,Scale=1}
\fi
% Use upquote if available, for straight quotes in verbatim environments
\IfFileExists{upquote.sty}{\usepackage{upquote}}{}
\IfFileExists{microtype.sty}{% use microtype if available
  \usepackage[]{microtype}
  \UseMicrotypeSet[protrusion]{basicmath} % disable protrusion for tt fonts
}{}
\makeatletter
\@ifundefined{KOMAClassName}{% if non-KOMA class
  \IfFileExists{parskip.sty}{%
    \usepackage{parskip}
  }{% else
    \setlength{\parindent}{0pt}
    \setlength{\parskip}{6pt plus 2pt minus 1pt}}
}{% if KOMA class
  \KOMAoptions{parskip=half}}
\makeatother
\usepackage{xcolor}
\IfFileExists{xurl.sty}{\usepackage{xurl}}{} % add URL line breaks if available
\IfFileExists{bookmark.sty}{\usepackage{bookmark}}{\usepackage{hyperref}}
\hypersetup{
  pdftitle={The Titanic - A disaster reflecting social inequality},
  pdfauthor={Damien Dupré},
  hidelinks,
  pdfcreator={LaTeX via pandoc}}
\urlstyle{same} % disable monospaced font for URLs
\usepackage[margin=1in]{geometry}
\usepackage{graphicx}
\makeatletter
\def\maxwidth{\ifdim\Gin@nat@width>\linewidth\linewidth\else\Gin@nat@width\fi}
\def\maxheight{\ifdim\Gin@nat@height>\textheight\textheight\else\Gin@nat@height\fi}
\makeatother
% Scale images if necessary, so that they will not overflow the page
% margins by default, and it is still possible to overwrite the defaults
% using explicit options in \includegraphics[width, height, ...]{}
\setkeys{Gin}{width=\maxwidth,height=\maxheight,keepaspectratio}
% Set default figure placement to htbp
\makeatletter
\def\fps@figure{htbp}
\makeatother
\setlength{\emergencystretch}{3em} % prevent overfull lines
\providecommand{\tightlist}{%
  \setlength{\itemsep}{0pt}\setlength{\parskip}{0pt}}
\setcounter{secnumdepth}{-\maxdimen} % remove section numbering
\usepackage{booktabs}
\usepackage{longtable}
\usepackage{array}
\usepackage{multirow}
\usepackage{wrapfig}
\usepackage{float}
\usepackage{colortbl}
\usepackage{pdflscape}
\usepackage{tabu}
\usepackage{threeparttable}
\usepackage{threeparttablex}
\usepackage[normalem]{ulem}
\usepackage{makecell}
\usepackage{xcolor}
\ifluatex
  \usepackage{selnolig}  % disable illegal ligatures
\fi

\title{The Titanic - A disaster reflecting social inequality}
\author{Damien Dupré}
\date{}

\begin{document}
\maketitle

Abstract This document is an example of what can be done using Rmarkdown
file to produce an academic/professional report. It focuses on the
titanic dataset which is one of the most classic dataset to practice
data analytic and data science. Indeed the catastrophe of the titanic is
not only interesting from a data point of view but also from an human
point of view. While the James Cameron movie did not focus too much on
the influence of classes on the final outcome, it appears that
passengers were not equals in age and gender. For these reasons, this
report will investigate what are the influence of age and gender on
passengers' ticket fare. Literature Review According to Saidatunnisa,
Sili and Nasrullah (2019), the inequality issues towards women existed
in one of the most outstanding movies of all times, Titanic. Titanic
movie began with the 84 years old woman named Rose DeWitt Bukater who
told about Titanic's tragic story on its first and last voyage April
1912. Rose DeWitt Bukater was regardless of her status as the upper
class, also a woman who experienced the inequality issues. Rose as a
woman had to be obedient towards the male supremacy society she lived
in. In addition, Researchers interested in inequalities in healthcare
wont to quote data from the Titanic disaster'' (Marmot 1999, p.~16).
This ritual stems from Antonovsky (1967), who opened his classic paper
with an illustration based on data from the Titanic, highlighting the
higher survival rate of the first- and second-class passengers, versus
the second- and third-class passengers, to demonstrate that social class
influences mortality (Pearson, 2008). Hypothesis 1: The average tickets
fare of female passengers is higher than the average ticket fare of male
passengers. Hypothesis 2: Passengers' tickets fare increases when their
age increases. Method The method section usually describes the
participants where the data was collected, the material and the
procedure used to collect the data, and the analysis performed to test
the hypotheses. Participants The data were collected from 1309
passengers known for having embarked on the Titanic. The average age of
passengers was 29.9 years old (SD = 14.4) but the age from 263
passengers is unknown (Table 1). n missing m\_age sd\_age 1309 263
29.88114 14.41349 Table 1. Passengers' demographics. From them, 466
where female passengers and 843 were male passengers (Table 2). Sex n
female 466 male 843 Table 2. Distribution of passenger gender. Material
Little is know over how these data were collected but it lists the
passengers name, age, gender, ticket class, number of siblings / spouses
aboard the Titanic, number of parents / children aboard the Titanic,
ticket number, passenger fare, cabin number, port of Embarkation (C =
Cherbourg, Q = Queenstown, S = Southampton) and ultimately their
survival (0 = No, 1 = Yes). Procedure This section usually describes how
the data were collected but here they were found online on the Kaggle
website \url{https://www.kaggle.com/c/titanic/data}. Data Analyses To
test these hypotheses, the following linear regression model will be
used: Fare=\textbackslash β\_0+\textbackslash β\_1
Gender+\textbackslash β\_2 Age+\textbackslash ϵ This model uses Fare as
a continuous outcome variable. Age is a continuous predictor and Gender
a categorical predictor with two categories (i.e., female and male). Age
and Gender are tested in the model as a main effect as indicated in
Figure 1.

Figure 1. Representation of the model tested. Results The average fare
for female passengers is slightly higher than the average fare for male
passengers (£46.2 for female passengers vs.~£26.2 for male passengers)
as show by the Figure 2.

Figure 2. Average ticket fare for female and for male passengers. It
also appears that the ticket fare increases when age increases as well
as shown in Figure 3.

Figure 3. Relationship between passenger's age and ticket fare. The
model tested is statistically significant in term of prediction of
passengers fare (R2 = 0.07, F(2, 1042) = 40.26, p \textless{} .001). The
effect of Age is statistically significant (b = 0.74, 95\% CI {[}0.51,
0.97{]}, t(1042) = 6.39, p \textless{} .001). The effect of Sex is
statistically significant as well (b = -23.04, 95\% CI {[}-29.81,
-16.27{]}, t(1042) = -6.68, p \textless{} .001). Discussion and
Conclusion An obvious reason why age of passengers influenced their
ticket fare is due to negotiated rates for children being cheaper than
adults. There is also the fact that older passengers prefer more
comfortable cabins and therefore will pay a higher price. However, the
reason why female passengers paid a higher price on average is still a
mystery. Probably, male passengers were satisfied with more simple
cabins. However, female passengers were usually travelling with their
children and therefore required a larger cabin. This explanation provide
supports for a latent gender inequality in western society. Female have
much more side expenses than males, this phenomenon is known as pink tax
(Lafferty, 2019) or tampon tax (Bennett, 2017). References Antonovsky,
A. (1967). Social class, life expectancy, and overall mortality. Milbank
Quarterly, 45(2), 31--73. Bennett, J. (2017). The tampon tax: Sales tax,
menstrual hygiene products, and necessity exemptions. Business
Entrepreneurship \& Tax Law Review, 1, 183. Lafferty, M. (2019). The
pink tax: The persistence of gender price disparity. Midwest Journal of
Undergrad Research, 11, 56-72. Marmot, M. G. (1999). Epidemiology of
socioeconomic status and health: Are determinants within countries the
same as between countries? Annals of the New York Academy of Sciences,
896, 16--29. Pearson, J. A. (2008). Can't buy me whiteness: New lessons
from the Titanic on race, ethnicity, and health. Du Bois Review: Social
Science Research on Race, 5(1), 27-47. Saidatunnisa, N. E., Sili, S., \&
Nasrullah, N. (2019). The inequality issues of male supremacy towards
rose character in titanic movie. Ilmu Budaya: Jurnal Bahasa, Sastra,
Seni dan Budaya, 3(4), 393-403.

\end{document}
